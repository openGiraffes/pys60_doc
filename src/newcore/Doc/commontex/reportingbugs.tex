\label{reporting-bugs}

Python is a mature programming language which has established a
reputation for stability.  In order to maintain this reputation, the
developers would like to know of any deficiencies you find in Python
or its documentation.

Before submitting a report, you will be required to log into SourceForge;
this will make it possible for the developers to contact you
for additional information if needed.  It is not possible to submit a
bug report anonymously.

All bug reports should be submitted via the Python Bug Tracker at
(\url{http://bugs.python.org}).  The
bug tracker offers a Web form which allows pertinent information to be
entered and submitted to the developers.

The first step in filing a report is to determine whether the problem
has already been reported.  The advantage in doing so, aside from
saving the developers time, is that you learn what has been done to
fix it; it may be that the problem has already been fixed for the next
release, or additional information is needed (in which case you are
welcome to provide it if you can!).  To do this, search the bug
database using the search box on the top side of the page.

If the problem you're reporting is not already in the bug tracker, go
back to the Python Bug Tracker.  Select the
``Create new'' link at the left of the page to open the bug reporting
form.

The submission form has a number of fields.  The only fields that are
required are the ``Title'' and ``Type'' fields.  For the title,
enter a \emph{very} short description of the problem; less than ten
words is good.  In the ``Change Note'' field, describe the problem in detail,
including what you expected to happen and what did happen.  Be sure to
include the version of Python you used, whether any extension modules
were involved, and what hardware and software platform you were using
(including version information as appropriate).

The only other field that you may want to set is the ``Components''
field, which allows you to place the bug report into broad categories
(such as ``Documentation'' or ``Library'').

Each bug report will be assigned to a developer who will determine
what needs to be done to correct the problem.  You will
receive an update each time action is taken on the bug.


\begin{seealso}
  \seetitle[http://www-mice.cs.ucl.ac.uk/multimedia/software/documentation/ReportingBugs.html]{How
        to Report Bugs Effectively}{Article which goes into some
        detail about how to create a useful bug report.  This
        describes what kind of information is useful and why it is
        useful.}

  \seetitle[http://www.mozilla.org/quality/bug-writing-guidelines.html]{Bug
        Writing Guidelines}{Information about writing a good bug
        report.  Some of this is specific to the Mozilla project, but
        describes general good practices.}
\end{seealso}
